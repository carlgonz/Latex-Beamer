%%%%%%%%%%%%%%%%%%%%%%%%%%%%%%%%%%%%%%%%%%%%%%%%%%%%%%%%%%%%%%
%%%%		PLANTILLA LATEX PARA PRESENTACIONES
%%%%				LATEX BEAMER TEMPLATE
%%%%	
%%%%	Autor	: Carlos Gonzalez Cortes
%%%%	Correo	: carlgonz@ug.uchile.cl
%%%%	Version	: 1.0
%%%%
%%%%	Notas	: Este codigo se entrega tal cual es y sin
%%%%			  ningun tipo de garantia. Sientase libre de
%%%%			  modificar y compartir.(acentos omitidos en
%%%%			  los comentarios por compatibilidad)
%%%%
%%%%%%%%%%%%%%%%%%%%%%%%%%%%%%%%%%%%%%%%%%%%%%%%%%%%%%%%%%%%%%


\documentclass[10pt]{beamer}
\usetheme{Frankfurt}

\usepackage[spanish]{babel}
%\usepackage[ansinew]{inputenc}
\usepackage[utf8]{inputenc}
%\usepackage[latin1]{inputenc}

\usepackage{url}

\usepackage{listings} %Para codigo fuente
\lstset{language=Matlab, numbers=left, numberstyle=\tiny, tabsize=4,framexleftmargin=5mm, basicstyle=\small,breaklines=true}


% --------------- ---------DATOS --------------------------------------------
\title{Titulo de la presentación}
\author[Autor]{Autor}
\institute[Universidad de Chile]{Universidad de Chile}
\logo{\includegraphics[scale=0.05]{img/escudoU.pdf}}
\date[\today]{}

\begin{document}

% --------------- ---------PORTADA --------------------------------------------

\begin{frame}
	\titlepage %Titulo
	
	%Una imagen en la portada
	\begin{center}
		\includegraphics[scale=0.2]{img/fcfm.png}
	\end{center}
\end{frame}

% --------------- ---------INDICE --------------------------------------------

\begin{frame}{Contenido}
\tableofcontents
\end{frame}

% ------------------------ SLIDE 1 --------------------------------------------

%Construyendo el indice
\section{Entrada en índice, sección 1 }
\subsection{Entrada en índice, subsección 1}

%Titulo de la diapositiva
\begin{frame}{Diapositiva 1: Ejemplos}
	
	% Agregar un bloque
	\begin{block}{Un bloque}
		Texto del bloque
	\end{block}
	
	%Agregar una lista
	\vspace{1cm} %Un espacio vertical
	Una lista
	\begin{itemize}
		\item Item 1
		\item Item 2
		\item Item 3
	\end{itemize}

\end{frame}

% ------------------------ SLIDE 2 --------------------------------------------
%Construyendo el indice
\subsection{Entrada en indice, sección 2 }

%Titulo de la diapositiva
\begin{frame}{Diapositiva 2: Imágenes}

%Usando un EXAMPLE-BLOCK
	\begin{exampleblock}{Esto es un ExampleBlock}
	El escudo de la Universidad de Chile\\
	
%Agregando una IMAGEN
	\begin{center}
		\includegraphics[scale=0.08]{img/escudoU.pdf}
	\end{center}
	\end{exampleblock}
\end{frame}

% ------------------------ SLIDE 3 --------------------------------------------
%Indice
\section{Sección 3}

%Titulo de la diapositiva
\begin{frame}{Dispositiva 3: Animación}

%Ejemplo de animacion, usar <+->
	\begin{exampleblock}{Primer bloque}<+->
		Ejemplo de animación
	\end{exampleblock}
	
	\begin{exampleblock}{Segundo bloque}<+->
		Ejemplo de animación
	\end{exampleblock}
	
	\begin{exampleblock}{Tercer bloque}<+->
		Ejemplo de animación
	\end{exampleblock}
\end{frame}

% ------------------------ SLIDE 4 --------------------------------------------
%Indice
\section{Sección 4}

%Titulo de la diapositiva
\begin{frame}{Diapositiva 4: Dos columnas}

\begin{center}
	Una diapositiva puede ser dividida en varias columnas
\end{center}

%Una diapositiva con dos columnas
	\begin{columns}
		%Columna Izquierda
		\column[t]{6cm}
			Columna izquierda, contiene una lista
			\begin{enumerate}
				\item Item 1
				\item Item 2
				\item Item 3
			\end{enumerate}

		%Columna Derecha
		\column[t]{6cm}
			Columna derecha, contiene una imagen
			\begin{center}
				\includegraphics[scale=0.2]{img/fcfm.png}
			\end{center}
	\end{columns}
\end{frame}


% ------------------------ SLIDE CC --------------------------------------------
% Entrada del indice
\section{CC}

%Diapositiva sin titulo
\begin{frame}
	\begin{center} \includegraphics[scale=1]{img/cc.png} \end{center}
	\vspace*{1cm}
	\begin{center}
	\textbf{Plantilla para presentaciones en Latex-Beamer por Carlos González Cortés} se encuentra bajo una Licencia Creative Commons Atribución-NoComercial-LicenciarIgual 3.0 Unported. \end{center}
\end{frame}


\end{document} 

% ============================= FRAME ==============================%
%\begin{frame}{Title}
%	\begin{columns}
%		\column[t]{6cm}
%		\column[t]{6cm}
%	\end{columns}
%\end{frame}
% ===================================================================%


% % ============================= IMAGEN ==============================%
% \texttt{\begin{figure}[ht!]
% \sf
% \centering
% \fbox{\includegraphics[scale=1.0]{img/circuito_segundo_diseno.jpg}}
% \caption{Circuito del segundo diseño}\label{circt_segundo_dis}
% \end{figure}}
% % ===================================================================%
% 
% % ============================= CODIGO ==============================%
% \begin{lstlisting}[caption={\sf codigo},label=codigo,frame=trBL]
% 
% \end{lstlisting}
% % ===================================================================%
%\lstinputlisting[caption={\sf codigo},label=codigo,frame=trBL]{sourceCode/HelloWorld.java}